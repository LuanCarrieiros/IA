\documentclass[sigconf]{acmart}

\usepackage[utf8]{inputenc}
\usepackage{listings}
\usepackage{color}
\usepackage{xcolor}
\usepackage{caption}
\usepackage{float}
\usepackage{graphicx}
\usepackage{url}
\usepackage{etoolbox}

\lstset{
    basicstyle=\ttfamily\footnotesize,
    breaklines=true,
    captionpos=b,
    numbers=left,
    numberstyle=\tiny,
    stepnumber=1,
    numbersep=5pt,
    frame=single,
    tabsize=4,
    showspaces=false,
    showstringspaces=false,
    showtabs=false,
    language=Python,
    float=H,
    abovecaptionskip=5pt,
    belowcaptionskip=5pt,
    keywordstyle=\color{blue},
    stringstyle=\color{red},
    commentstyle=\color{gray}
}

\title{Resolução do 8-Puzzle com Métodos de Busca: BFS, DFS e A*}

\author{Luan Barbosa Rosa Carrieiros}
\affiliation{
  \institution{PUC Minas}
  \city{Belo Horizonte}
  \state{MG}
  \country{Brasil}
}
\email{luan.rosa@sga.pucminas.br}

\begin{document}

\begin{abstract}
Este artigo apresenta a implementação do jogo 8-Puzzle utilizando três métodos de busca: busca em largura (BFS), busca em profundidade (DFS) e o algoritmo A* com duas heurísticas diferentes. A interface gráfica em Tkinter foi adicionada para visualização interativa. O objetivo é comparar o desempenho entre os algoritmos quanto à eficiência, tempo de execução, profundidade da solução e número de nós visitados.
\end{abstract}

\keywords{Inteligência Artificial, 8-puzzle, Busca em Largura, Busca em Profundidade, A*, Heurísticas, Tkinter}

\maketitle

\section{Introdução}

O 8-Puzzle é um clássico problema de busca em Inteligência Artificial. Neste trabalho, implementamos três algoritmos de solução (BFS, DFS e A*) e adicionamos uma interface em Tkinter para visualização e controle interativos.

\section{Métodos de Busca e Funções Auxiliares}

\subsection{Geração de Vizinhos}
\begin{figure}[H]
\centering
\begin{lstlisting}[caption={Função get\_neighbors}]
def get_neighbors(state):
    neighbors = []
    index = state.index(0)
    row, col = divmod(index, 3)
    moves = [
        ("Cima", index-3, row>0),
        ("Baixo", index+3, row<2),
        ("Esquerda", index-1, col>0),
        ("Direita", index+1, col<2)
    ]
    for direction,new_idx,valid in moves:
        if valid:
            new_state = list(state)
            new_state[index], new_state[new_idx] = new_state[new_idx], new_state[index]
            neighbors.append((new_state,direction))
    return neighbors
\end{lstlisting}
\end{figure}

\subsection{Heurísticas}
\begin{figure}[H]
\centering
\begin{lstlisting}[caption={Heurística Manhattan}]
def manhattan(state):
    dist = 0
    for i,val in enumerate(state):
        if val==0: continue
        goal_r,goal_c = divmod(val-1,3)
        r,c = divmod(i,3)
        dist += abs(goal_r-r)+abs(goal_c-c)
    return dist
\end{lstlisting}
\begin{lstlisting}[caption={Heurística Peças Fora do Lugar}]
def heuristic_misplaced(state):
    goal = list(range(1,9))+[0]
    return sum(1 for i,val in enumerate(state)
               if val!=0 and val!=goal[i])
\end{lstlisting}
\end{figure}

\subsection{Busca em Largura (BFS)}
\begin{figure}[H]
\centering
\begin{lstlisting}[caption={Implementação BFS}]
def bfs(start):
    goal=list(range(1,9))+[0]
    queue=deque([(start,[])])
    visited=set()
    while queue:
        state,path=queue.popleft()
        if state==goal:
            return path
        visited.add(tuple(state))
        for nbr,dir in get_neighbors(state):
            if tuple(nbr) not in visited:
                queue.append((nbr,path+[dir]))
    return []
\end{lstlisting}
\end{figure}

\subsection{Busca em Profundidade Limitada (DFS)}
\begin{figure}[H]
\centering
\begin{lstlisting}[caption={Implementação DFS com limite}] 
def dfs_limited(start,limit=30):
    goal=list(range(1,9))+[0]; visited=set()
    def rec(s,p,d):
        if s==goal: return p
        if d>=limit: return None
        visited.add(tuple(s))
        for nbr,dir in get_neighbors(s):
            t=tuple(nbr)
            if t not in visited:
                res=rec(nbr,p+[dir],d+1)
                if res is not None: return res
        return None
    return rec(start,[],0) or []
def dfs(start): return dfs_limited(start)
\end{lstlisting}
\end{figure}

\subsection{Busca A*}
\begin{figure}[H]
\centering
\begin{lstlisting}[caption={Implementação A* (Manhattan)}]
def astar_manhattan(start):
    goal=list(range(1,9))+[0]
    heap=[(manhattan(start),0,start,[])]
    visited=set()
    while heap:
        _,g,s,path=heapq.heappop(heap)
        if s==goal: return path
        visited.add(tuple(s))
        for nbr,dir in get_neighbors(s):
            if tuple(nbr) not in visited:
                heapq.heappush(heap,(g+1+manhattan(nbr),g+1,nbr,path+[dir]))
    return []
\end{lstlisting}
\begin{lstlisting}[caption={Implementação A* (Peças Fora)}]
def astar_misplaced(start):
    goal=list(range(1,9))+[0]
    heap=[(heuristic_misplaced(start),0,start,[])]
    visited=set()
    while heap:
        _,g,s,path=heapq.heappop(heap)
        if s==goal: return path
        visited.add(tuple(s))
        for nbr,dir in get_neighbors(s):
            if tuple(nbr) not in visited:
                heapq.heappush(heap,(g+1+heuristic_misplaced(nbr),g+1,nbr,path+[dir]))
    return []
\end{lstlisting}
\end{figure}

\section{Interface Gráfica (Tkinter)}
\begin{figure}[H]
\centering
\begin{lstlisting}[caption={Código da interface}] 
# (ver arquivo puzzle.py)
if __name__=='__main__':
    root=tk.Tk()
    app=PuzzleApp(root)
    root.mainloop()
\end{lstlisting}
\caption*{Interface completa disponível em \texttt{puzzle.py}.}
\end{figure}

\section{Comparativo de Desempenho}
\begin{table}[H]
\centering
\caption{Desempenho dos Algoritmos}
\resizebox{\columnwidth}{!}{
\begin{tabular}{|l|c|c|c|l|}
\hline
Método & Tempo (s) & Nós Visitados & Movimentos & Heurística \\
\hline
BFS & 0.38 & 60 914 & 21 & — \\
DFS limitado & — & — & — & — \\
A* (Manhattan) & 0.0012 & 133 & 21 & Manhattan \\
A* (Peças Fora) & 0.1062 & 4 822 & 21 & Fora do lugar \\
\hline
\end{tabular}
}
\end{table}

\section{Conclusão}

A análise comparativa demonstrou que o algoritmo A* com heurística de Manhattan apresentou o melhor desempenho geral, com menor tempo e menor número de nós visitados. A heurística de peças fora do lugar também teve desempenho satisfatório. A BFS, embora completa, foi significativamente mais lenta. A DFS não conseguiu encontrar solução no limite de profundidade adotado. Conclui-se que heurísticas eficazes tornam o A* a melhor abordagem para resolver o 8-Puzzle.

\section{Código desenvolvido}
\textbf{Link para o código e executável:} \url{https://github.com/LuanCarrieiros/IA/tree/main/Lista%20Extra}

\end{document}
